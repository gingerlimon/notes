\chapter{Introduction to Proofs}

\section{What are Proofs?}

\begin{defn}
  A \textit{proof} is a mathematical argument which shows that the conclusion logically follow
  from the stated assumptions.
\end{defn}

A proof is said to be more rigorous if it gives more justification to the steps in the argument.
However, overly rigourous proofs might disract the audience from the important points. Proofs
can be thought of as being deistriubuted in a \textit{magic-motivated} axis.

In \textit{magical proofs}, it appears as if one pulled out steps from thin air. Initial steps
appear unmotivated, and it might me difficult to remember. \textit{Motivated proofs} flow
naturally. They follow the thought process of the brain-storm. But they tend to be logically
more complex than \textit{magical proofs}. To keep the proof simple while also being resaonable, it is usually a good compromise to
introduce the idea of the proof nonrigorously, before the body of the proof.

One can also think of a \textit{elaboration-condensation} axis. While more \textit{elaborated}
proofs are more rigorous, they can be tiring and logically complex. Highly \textit{condensed}
proofs are hard to follow, and tend to reduce necessary steps. The extent to which
a proof should be elaborated or condensed depend on the target audience. Things that are obvious
to experienced mathematicians might not be so for an undergradute.

\section{Some Simple Proofs}

To demostrate the concepts, we prove some propositions related to big numbers.

\begin{prop}
  $10^6<2^{20}<10^7$
\end{prop}

\begin{proof}
\end{proof}
